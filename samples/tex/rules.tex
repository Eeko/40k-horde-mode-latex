\documentclass{HordeModeTarot}

\setTarotCardGeometry
%\printRulesTrimmingLines

\title{Warhammer 40k Horde Mode rules}
\author{Poorhammer Podcast}
\date{\today \\v0.83}

\setlength{\parskip}{3pt}
\fontsize{10pt}{10pt}
\selectfont
\begin{document}

\maketitle

\section{Army Construction}
In 40K Horde Mode you first must decide on a game size, 1,000 points or 2,000 points.  Once a point count has been chosen you must then choose the number of players.

(Technically, any number of players should work, but the game is designed for 1-4 players. Optionally 1 extra player could be added as a dedicated Game Master controlling the Horde and helping with resolving steps of the game, however this role is not necessary.)

A horde faction should also be chosen at this point, but no list needs to be created for it.  

Designer’s Note: A horde can use multiple factions for their spawning tables in order to round out spawning tables for your playgroup’s collection. Playing with multiple factions in the horde is advised when playing with either type of Knights, both of whom have several fail-cases on their spawn tables.

\begin{itemize}
\item Divide your total points equally among all players, rounded down. (ex. 3 players at 1,000 points each get 333 points.)  
Note that due to resource acquisition more players tends to lead to easier games, artificially lowering your available point total may help counteract this if you still want a difficult experience with 4+ players.
\item Each player’s army must contain a warlord and be a valid list by 40k 10th edition matched play rules.
\item Additionally, no enhancements may be taken in these starting lists. Enhancements are purchased during the game.
\end{itemize}

\section{Table Setup}
Follow these steps to set up the table and begin the game:
\begin{enumerate}
\item Choose a deployment zone setup randomly from the 40K 10th Edition Leviathan Mission Pack.  For ease of use Dawn of War and Hammer and Anvil are the only two deployments recommended. Using other shaped deployment zones drastically increases the difficulty of finding the proper areas of Spawning Zones.
\item Once the deployment zones are decided and objectives are placed on the board, the Horde side is set as the Attacker and players as the Defender.
\item On the Horde deployment zone mark it into even area sections based on game size.  1,000 point games are split evenly into 2 spawn zones and 2,000 point games are split evenly into 4 spawn zones.  These zones should be placed as equal areas, making non-rectangular deployments much more difficult to calculate. It is recommended you place reminder tokens in the center of each spawning zone’s area.
\item The board should be set up with terrain only on the Horde side of the board and in No Man’s Land.
\item Place all player units on the board or in reserves following basic Matched Play Leviathan rules, skip placing anything for the horde.
\item Shuffle the Misery Deck, the Secret Objectives Deck, and Secondary Mission Deck, placing them to the side.
\item Choose if the game will be 5 rounds or infinite rounds.  5 rounds is the recommended number, but some players enjoy the play until you lose style of game.
\item If the 5 rounds game mode was chosen, shuffle the Secret Objective deck and deal two objectives to each player. They each choose one objective and return the others to the Secret Objective deck face down.  Shuffle the Secret Objective deck.  (If there are not enough Secret Objectives to deal two to each player due to a high player count game, deal 1 instead.)
\end{enumerate}

\section{Playing the Game}

\subsection{Warhammer 40 000 Base rules}
40K Horde Mode uses the base rules of the game with the following additions and alterations:

\begin{itemize}
\item Players move through phases together and activate units and abilities at the correct time in the phases, but in any order.
\item Players act as a single team, you count as friendly units to each other. Buffs that affect friendly units affect the units of other players if they have the correct keywords.
\item CP and RP are tracked at a personal level for each player.
\item Players and the Horde do not naturally generate CP in the Command Phase.
\item There is no cap on CP generation per turn.
\item The Horde always has the first turn.
\item The Horde does not benefit from Army or Detachment rules, only datasheet level rules.
\item An extra step to the Command Phase is added called the “Resupply Step” placed after the Battle-Shock Step.
\item Players may not reveal their Secret Objective cards to each other until the card tells them to or the game ends. Players may claim to have any secret objective, including telling the truth.
\end{itemize}

If for any reason the Misery Deck or Secondary Mission deck is empty, reshuffle all used cards into the deck.


\subsection{At the beginning of each Battle Round}
\begin{enumerate}
\item Remove any active Misery Cards to discard.
\item Reveal any cards triggered from the Misery Deck (including modifiers from failed Secondary Missions last round.)
\item The new Secondary Mission for the round is revealed.
\item All mission rewards that require resolving at the start of the battle round are resolved.
\item Spawn the Horde. (See Spawning the Horde.)
\item Any beginning of battle round effects resolve after the Horde is spawned. (Rad Cohort detachment rule and Leagues of Votann army rule as an example.)
\end{enumerate}

\subsection{At the beginning of Battle Rounds 3 and 4 }
\begin{itemize}
\item +1 to Spawning roll results. 
\item 1 Misery card is revealed for this battle round.
\end{itemize}

\subsection{At the beginning of Battle Round 5+ }
\begin{itemize}
\item +2 to Spawning roll results. 
\item 3 Misery cards are revealed for this battle round.
\end{itemize}

\subsection{End of the Battle Round}
\begin{itemize}
\item All revealed Secondary Missions are resolved, failed missions trigger their fail cases, successful missions give their listed reward. Misery card related results are added to the next Battle Round’s reveals.
\end{itemize}

\subsection{Reinforcement Points}
\begin{itemize}
\item At the battle shock step of the players’ command phase, when primary objectives would normally be scored, instead count up each objective marker controlled by any player.  For each objective marker owned by the players’ team each player receives 1RP.  (This occurs even during the first battle round.)
\item Any time a Horde unit is destroyed, if that destruction was due to a player’s attack or action, that player gains 1RP. 
\item RP is also rewarded from some successful secondary missions. Unless otherwise stated, each player receives the listed amount of RP rewarded from those secondary missions. The same is true for any RP lost from a failed secondary mission.
\item The minimum RP is 0.
\end{itemize}

\subsection{The Resupply Step }
During the Resupply Step of the Command Phase, which is added after the Battle-shock Step, players may then spend any RP they have built up on benefits from the Reinforcement Points table.  Once all players have finished spending their RP the step ends and the Command Phase is moved to Movement Phase as per usual.

\section{Spawning the Horde}
Spawning the Horde is done via the following steps: 
\begin{enumerate}
\item Choose a Horde Spawning Zone.
\item Destroy any player owned units and previously spawned Horde units that are within the zone. Take note of the cost of any Horde unit removed this way. 
Designer’s Note: This should be a rare occurrence in normal play, designed to stop the strategy of locking hordes in their deployment zone in hopes of stopping spawning of more powerful units.  
\item Roll 2D6 adding any modifiers to the result to find the Horde unit bracket to spawn into the zone. Any removed Horde unit’s bracket’s lowest roll value is added to the roll as an additional modifier. An unmodified roll of 2 is always a “No Spawn.”
\item Choose a unit from the bracket based on your available collection or choose randomly from the list. See Horde Spawning Tables for brackets and unit options. When selecting loadouts for the unit, use what you have available.  
Designer’s Note: When possible give the unit all relevant wargear and weapons, as brackets were determined using properly equipped units.
Designer’s Note: Remember that you may always pick to place two units from the bracket below the rolled bracket instead.
\item Place the chosen unit into the spawning zone. Spawn units as close to the front of the deployment zone as possible. If a unit will not fit in the spawn area due to base sizes, make the best possible attempt at fitting it within that section of deployment. Horde Units placed down may overflow into No Man’s Land if it is unavoidable. (It does not count as entering the battlefield this round for any rules purposes. Play it as though it has been in play since game start.)
\item Repeat the process for each of the remaining spawning zones.
\end{enumerate}

Designer’s Note: The first character spawned by the horde acts as the Warlord for the Horde army for all rules that mention an enemy Warlord. (Imperial Knights, etc.)
Designer’s Note: The mode is designed to work with any spare army your playgroup has access to, given it has a large amount of points available to pull from.  Ideally 1,000-2,000 more points than the chosen game size at minimum. Aircraft exist in the Spawn tables, but are recommended against for their issues with the AI in this game mode.  Choose them only as a last resort.  The same goes for Drop Pods, which are functionally useless for the Horde.

\section{The Horde “AI”}
Designer Note: This is an attempt to make a simple set of rules to pilot the Horde in a way that is at least on par with general zombie AI in video games.  We don’t want this overly complicated because it should feel intuitive and you should not need to check these rules after a game of piloting the Horde.

\subsection{The horde utilizes all datasheet abilities it can}
Certain datasheet abilities don’t do anything due to the lack of horde stratagem usage, army rules, and detachment rules. (Khal, Sisters of Battle, and Farseer have example abilities.)  These should be ignored.  All other datasheet abilities should be utilized at the appropriate time in the turn.  Abilities that require you to choose a mode (Belisarius Cawl, etc.) should be chosen randomly each time.


\subsection{During the movement phase each unit in the Horde attempts to move directly toward the following}
In order of importance.
\begin{enumerate}
\item The closest visible enemy. 
\item The closest visible objective not already owned by another Horde unit, ignoring the ones any objective in the Horde Spawning Areas.
\item The Defending Player board edge.
\end{enumerate}
Designer’s Note: This means some horde units will camp objectives if not baited away by visible units.

\subsection{Each unit moves if appropriate}
Exceptions are as follows:
\begin{itemize}
\item Units where the majority of models are equipped with weapons with the Heavy ability and the unit has an enemy unit visible within range of those weapons.
\item A single model unit whose primary ranged weapon has the Heavy ability and has an enemy unit visible within range of those weapons
\item Units that are within shooting range of a visible enemy unit and will lose some significant bonus from abilities if they move.
\item Units within the spawn area always move to avoid deletion even if other bullet points are true.
\end{itemize}
Designer’s Note: Move unless a unit has no reason to ruin their shooting bonus.

\subsection{If advancing is advantageous, it does so} 
Advancing is considered advantageous if it meets any of the following criteria:
\begin{itemize}
\item All weapons are assault 
\item The unit has the ability to advance and charge
\item The unit could not attempt a charge anyway and the only ranged weapons that it loses access to are “side-arms”
Side-arms are defined as: PISTOLs on units with melee loadouts beyond close combat weapons or ranged weapons on attached characters. (Orks should not slow down to shoot Sluggas.  Lychguard should not slow down to let an attached character shoot a Staff of Light. etc.)
\end{itemize}

\subsection{Units disembark from transports when appropriate}
\begin{itemize}
\item If there is a visible enemy unit within range of the ranged weapons and there is not enough Firing Deck X on the transport to fire all non-”side-arms” (see above.)
\item If disembarking would allow for an attempted charge for units that would attempt to charge this turn.
\item The unit will not disembark inside of a spawning zone even if the above are true.
\end{itemize}

\subsection{In the shooting phase each model shoots the closest legal target}
\begin{itemize}
\item Pick randomly if two are equally close.
\item If a model is equipped with an Anti-X weapon it attempts to shoot the closest visible X keyword unit, if none are visible within range, default back to the closest legal target.
\end{itemize}

\subsection{In the charge phase each unit always charges}
But only if the majority of the unit has a weapon with a profile better than Close Combat Weapon equivalent. 
Designer’s Note: This should stop things like tanks, guardsmen and T’au from charging.

\subsection{Horde units already in combat stay in combat} 
Unless they have Fallback and Shoot, Fallback and Charge, or Fallback, Shoot, and Charge. If they would not charge due to above rules they fall back a full move in this scenario, if they would charge again, only fall back 2”.

\subsection{Horde models in melee attack the closest enemy model’s unit if able}
Designer’s Note: This means sometimes horde units will split attacks on a model-by-model basis to get the most attacks in as possible.

\subsection{Horde units always attempt consolidation toward the closest enemy unit or objective marker if possible.}

\section{Winning the Game}
If the game is not being played in infinite mode, at the end of Battle Round 5 the game ends. All players with units still remaining reveal their hidden Secret Objective cards. All players who have revealed and completed their Secret Objective win the game, regardless of their survival. Players win the game individually.

\end{document}
